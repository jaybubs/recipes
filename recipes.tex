\documentclass[10pt]{article}
\usepackage{lmodern}
\usepackage[margin=0.5in]{geometry}
\usepackage{amsmath}
\usepackage{physics}
\usepackage{array}
\usepackage{cancel} 
\usepackage[bookmarks=true]{hyperref} 
\begin{document}
\section{Soups}%
\label{sec:soups}

\subsection{Goulash}%
\label{sub:goulash}
\subsubsection{Ingredients}%
\label{ssub:goulash_ingredients}
This list is scaled to be used on a massive 8 L pot (idk the actual volume, there should be leftover space between the cover and the good shit)
\begin{itemize}
	\item 250 g of butter (can always substitute half of that with lard for extra porkiness or coconut oil for extra sat fat deliciousness)	
	\item 1 kg of some delicious meat, recommendations:
		\begin{itemize}
			\item pork neck or collar
			\item pork shoulder
			\item pork belly
			\item beef equivalents
			\item basically any fatty cuts, like actually fatty, aim for at least 25\% fattiness, fo real
			\item and ofc any combinations
		\end{itemize}
	\item 2-3 large-ass juicy tomatoes - make sure those fuckers are sweet af
	\item 2-3 peppers
	\item 3-5 onions
	\item 1 garlic (a bulb, not just a clove)
	\item a goddamn kilo of carrots
	\item some smaller amounts (total up to 500 g) of other shit if available such as:
		\begin{itemize}
			\item parsnips
			\item tournip
			\item aubergine
			\item kohlrabi
			\item potatoes (make sure they taste good, not some peasant basic garbage that just adds carb like a dumbass)
		\end{itemize}
	\item warm beer (0.5 or 1 l if going full beer goulash) - lager, pale ale, stout, anything goes really!, wine (0.3 l) - red, not fortified, some sweeter whites also work, hard alcohol (0.1 l - careful, not everything works, herby liquors are great, vodka is dumb you cheap arse)
	\item water if needed
	\item herbs and spices and shit:
		\begin{itemize}
			\item salt
			\item pepper (black/green/red/white/mix)
			\item primary herb:
				\begin{itemize}
					\item basil
					\item oregano
					\item parsley
				\end{itemize}
			\item secondary herb:
				\begin{itemize}
					\item oregano
					\item dill
					\item rosemary
				\end{itemize}
			\item optional chillies, you decide on the strength:
				\begin{itemize}
					\item birds eye - excellent pairing
					\item cayenne
					\item scorpion
					\item habaneros
					\item carolina reapers
				\end{itemize}
			\item grated mozzarella
		\end{itemize}
	\item high quality beers or wine
\end{itemize}
\subsubsection{Procedure}%
\label{ssub:goulash_procedure}
Open a can of the high quality beer or the wine and start drinknig. If your knives ain't sharp now is a good time to give them some treatment before the alcohol kicks in.\par
Put the pot on your stove and set that fucker to minimum flame/heat. Chuck in the block of butter and let it slowly melt.\par
In the meantime prep your chopping board area and start with the onions - chop'em up into tinycubes and chuck'em into the melting butter, lid it, and gently increase the temperature.\par
Next crush the garlic bulb. At this point decide whether you wanna chop it up into tiny bits or just crush them with the side of the knife (preferred method). Bonus points if you keep the peel on for dat dere extra fibre (srs). Chuck that shit in.\par
While the onion/garlic combo is frying in the butter soup, start chopping up the meat into nice big cubey chunks. Once you're done check on the garlnion mixture - when it's a bit foamy and yellowy, that's the perfect time to chuck all the meat in.\par
Now grab the salt and pepper and grind that shit all over the delicious meat. Keep grinding, make sure it's really rich, don't be a goddamn wuss, more salt. And a bit more pepper ffs. Got it? Great, now a dib more of both and cover it up. Now Start a timer for 15 minutes.\par
Great job, take another big sip of your delicious beer/wine.\par
Drink faster cos now you got about 15 minutes to dice the tomatoes and peppers - feel free to go chunky or fine according to your preference. Set these aside. Once the timer goes wild check the meat and give it a big big big stir, then cover it up again and time another 5 minutes. Now if you were fast enjoy the next 5 minutes with more of your delicious beer/wine. Otherwise keep dicing ffs, you only have 5 minutes left. Jesus.\par
Time's up chuck that tomato and pepper shit in! If your alcoholic ingredient is hard liquor this is the time to chuck it in, otherwise hold back. Set the timer for 15 minutes and cover that sucker up.\par
In the meantime prep your herbs, grab a couple of soup spoons of your primary herb, a soup spoon of your secondary herb and whatever amount of chillies you think your arse can handle. 10 minutes left? keep drinking ffs\par
Timer went off? whachu waiting for, chuck those herbs in. And now is also the time you chuck in the beer or wine. Even if you're going for full beer goulash, chuck in only 0.5 l at this point. Set the timer for 45 minutes and lower the heat.\par
Now that you have plenty of time to drink, maybe open a second can or grab another full glass of wine. It's time to get dicing all the other veg, so go ahead. Once you're done, figure out what you want to watch, you should have at least 20 minutes to find a movie or a show or something.\par
Once time's up, chuck all that shit in and fill the pot up with water (or the other can of beer) so all the veg is covered. Stir that shit, and set a timer for 1 hr, no less.\par
If you're efficent you should have chosen a thing to watch by now, might as well start.\par
When time's up, taste that fucker and decide whether it needs any more (it shouldn't, but in case you fucked it up).\par
Serve that shit piping hot, bonus points for topping your bowl with grated mozzarella.\par
\par
Enjoy!


\section{breads}%
\label{sec:breads}

\subsection{Czech Bread}%
\label{sub:czech_bread}

\subsubsection{Ingredients}%
\label{ssub:czech_bread_ingredients}
This amount makes one standard sized loaf to serve 4 ppl in one sitting

\begin{itemize}
	\item 500 g flour
	\item 300 ml room temp full fat milk
	\item 1 packet of yeast
	\item 1 tbl spoon of sweetener (sucralose/stevia) or 2 tbl spaces of sugar
	\item 1 large onion
	\item a chunk of butter, idk how much, i just cut, the more the better really
	\item 1 bottle of Pilsner Urquell
\end{itemize}
\subsubsection{Procedure}%
\label{ssub:czech_bread_procedure}

Open the bottle of PU and start drinking. Find a mug, a bowl, a small pot, frying pan and a chopping board you'll need those.\par

Ditch the hopefully warmed up milk into the mug and chuck the yeast on top, stirr a bit and let sit. Chuck the flour into a large bowl, and chuck the butter into a tiny pot and melt it. Don't actually let it burn. When it's \textit{just} melted, take it off the heat and take a big large sip of PU, you deserve it.\par

Get your chopping board, and start chopping up the onion into tiny little cubelettes. Chuck another chunk of butter into a frying pan, and get frying dat delicious onion. When it's just foamed up and slightly brown, take it off the heat. Take another gulp of beer.\par

By now the yeast should of worked its might and magic, and the melted butter in your other pan cooled down a bit, so chuck both guys into the bowl with flour, grab yourself a wooden spoon and get stirring, making sure all of the flour has been covered and a neat sticky dough has started to form.\par

Some minutes in, the onion has hopefully cooled down, so chuck that shit into the dough. At this point you better take a few big gulps of your beer, cos you gonna get your hands dirty boy. Work those onions into the dough, and knead it until it's either just or no longer sticking to the sides of the bowl. When it does, grab some extra flour and chuck it onto your chopping board, and transfer the dough onto it. Now you can knead it extra hard.\par

After a couple of minutes of some heavy kneading, either clean the bowl, or get another one. Grab a little chunk of butter (or just uravel the pack a bit) and smear it over the sides of the bowl. Once done, transfer the kneaded dough back into the bowl. Cover it with a dish cloth and let it sit in some place that's warm.\par

Now you gonna wait, so you might as well finish that beer. At 30 degrees C, the dough needs about an hour - an hour and half to rise like lazarus. At lower temps you might have to wait substantially longer. Do not leave it in a place above 35 you brute.\par

You managed to wait? great, check the dough, it should have \textit{at least} doubled up in size, good job! feel free to tear into it and give it a taste, it should taste tasty, like a tasty dough sohuld taste.\par

Turn your oven on, onto the fan function, and set the temp to 190 C.\par

Prep yourself a baking tray, cover the bottom in butter (same method as for the bowl above) and transfer the loaf onto it. If you're feeling insecure or want a slightly denser texture, feel free to knead the dough a bit, but let it rise another 20-30 minutes again (in the tray). Make some fancy incisions across the top of the loaf so it can bloom a bit in the oven. \par

In a second baking tray (or some heat resistant cup/mug/bowl/sacrificial vessel) put like 200ml of water and leave it at the bottom of the oven. When the oven reaches 190, place the tray with bread in the middle, and set the timer for 90 minutes.\par

Uhh...eat?
\end{document}
