\documentclass[10pt]{article}
\usepackage{lmodern}
\usepackage[margin=0.5in]{geometry}
\usepackage{amsmath}
\usepackage{physics}
\usepackage{array}
\usepackage{cancel} 
\usepackage[bookmarks=true]{hyperref} 
\begin{document}
\section{Soups and stews}%
\label{sec:soups}

\subsection{Goulash}%
\label{sub:goulash}
\subsubsection{Ingredients}%
\label{ssub:goulash_ingredients}
This list is scaled to be used on a massive 8 L pot (idk the actual volume, there should be leftover space between the cover and the good shit)
\begin{itemize}
	\item 250 g of butter (can always substitute half of that with lard for extra porkiness or coconut oil for extra sat fat deliciousness)	
	\item 1 kg of some delicious meat, recommendations:
		\begin{itemize}
			\item pork neck or collar
			\item pork shoulder
			\item pork belly
			\item beef equivalents
			\item basically any fatty cuts, like actually fatty, aim for at least 25\% fattiness, fo real
			\item and ofc any combinations
		\end{itemize}
	\item 2-3 large-ass juicy tomatoes - make sure those fuckers are sweet af
	\item 2-3 peppers
	\item 3-5 onions
	\item 1 garlic (a bulb, not just a clove)
	\item a goddamn kilo of carrots
	\item some smaller amounts (total up to 500 g) of other shit if available such as:
		\begin{itemize}
			\item parsnips
			\item tournip
			\item aubergine
			\item kohlrabi
			\item potatoes (make sure they taste good, not some peasant basic garbage that just adds carb like a dumbass)
		\end{itemize}
	\item warm beer (0.5 or 1 l if going full beer goulash) - lager, pale ale, stout, anything goes really!, wine (0.3 l) - red, not fortified, some sweeter whites also work, hard alcohol (0.1 l - careful, not everything works, herby liquors are great, vodka is dumb you cheap arse)
	\item water if needed
	\item herbs and spices and shit:
		\begin{itemize}
			\item salt
			\item pepper (black/green/red/white/mix)
			\item primary herb:
				\begin{itemize}
					\item basil
					\item oregano
					\item parsley
				\end{itemize}
			\item secondary herb:
				\begin{itemize}
					\item oregano
					\item dill
					\item rosemary
				\end{itemize}
			\item optional chillies, you decide on the strength:
				\begin{itemize}
					\item birds eye - excellent pairing
					\item cayenne
					\item scorpion
					\item habaneros
					\item carolina reapers
				\end{itemize}
			\item grated mozzarella
		\end{itemize}
	\item high quality beers or wine
\end{itemize}
\subsubsection{Procedure}%
\label{ssub:goulash_procedure}
Open a can of the high quality beer or the wine and start drinknig. If your knives ain't sharp now is a good time to give them some treatment before the alcohol kicks in.\par
Put the pot on your stove and set that fucker to minimum flame/heat. Chuck in the block of butter and let it slowly melt.\par
In the meantime prep your chopping board area and start with the onions - chop'em up into tinycubes and chuck'em into the melting butter, lid it, and gently increase the temperature.\par
Next crush the garlic bulb. At this point decide whether you wanna chop it up into tiny bits or just crush them with the side of the knife (preferred method). Bonus points if you keep the peel on for dat dere extra fibre (srs). Chuck that shit in.\par
While the onion/garlic combo is frying in the butter soup, start chopping up the meat into nice big cubey chunks. Once you're done check on the garlnion mixture - when it's a bit foamy and yellowy, that's the perfect time to chuck all the meat in.\par
Now grab the salt and pepper and grind that shit all over the delicious meat. Keep grinding, make sure it's really rich, don't be a goddamn wuss, more salt. And a bit more pepper ffs. Got it? Great, now a dib more of both and cover it up. Now Start a timer for 15 minutes.\par
Great job, take another big sip of your delicious beer/wine.\par
Drink faster cos now you got about 15 minutes to dice the tomatoes and peppers - feel free to go chunky or fine according to your preference. Set these aside. Once the timer goes wild check the meat and give it a big big big stir, then cover it up again and time another 5 minutes. Now if you were fast enjoy the next 5 minutes with more of your delicious beer/wine. Otherwise keep dicing ffs, you only have 5 minutes left. Jesus.\par
Time's up chuck that tomato and pepper shit in! If your alcoholic ingredient is hard liquor this is the time to chuck it in, otherwise hold back. Set the timer for 15 minutes and cover that sucker up.\par
In the meantime prep your herbs, grab a couple of soup spoons of your primary herb, a soup spoon of your secondary herb and whatever amount of chillies you think your arse can handle. 10 minutes left? keep drinking ffs\par
Timer went off? whachu waiting for, chuck those herbs in. And now is also the time you chuck in the beer or wine. Even if you're going for full beer goulash, chuck in only 0.5 l at this point. Set the timer for 45 minutes and lower the heat.\par
Now that you have plenty of time to drink, maybe open a second can or grab another full glass of wine. It's time to get dicing all the other veg, so go ahead. Once you're done, figure out what you want to watch, you should have at least 20 minutes to find a movie or a show or something.\par
Once time's up, chuck all that shit in and fill the pot up with water (or the other can of beer) so all the veg is covered. Stir that shit, and set a timer for 1 hr, no less.\par
If you're efficent you should have chosen a thing to watch by now, might as well start.\par
When time's up, taste that fucker and decide whether it needs any more (it shouldn't, but in case you fucked it up).\par
Serve that shit piping hot, bonus points for topping your bowl with grated mozzarella.\par
\par
Enjoy!


\subsection{Perkelt}%
\label{sub:perkelt}

\subsubsection{Ingredients}%
\label{ssub:ingredients}

\begin{itemize}
	\item 500g pork collar
	\item 2 medium onions
	\item 3 heaped tablespoons of paprika
	\item 50g lard or butter
	\item 330ml beer
	\item (optional) 100ml sour cream
	\item (optional) 250g sauerkraut
\end{itemize}
\subsubsection{Procedure}%
\label{ssub:procedure}
Fine chop your onions, chuck them with butter and paprika into a skillet and get frying. When the onions get glassy, pour about a third of the beer and let it simmer.\par

While shit's simmering, dice your pork just don't forget to occasionally stir the stuff in the pan. \par

Add the pork when most of the beer has reduced, stirr it in and add another third of the beer and just keep simmering. Repeat the entire cycle once the beer has simmered out. If you have opten for any optional ingredients, add them during this last cycle. The stew is ready when the contents have become thick and saucy.\par

Recommended with halusky \ref{sub:halusky}.


\section{Breads}%
\label{sec:breads}

\subsection{Classic Bread}%
\label{sub:classic_bread}

\subsubsection{Ingredients}%
\label{ssub:classic_bread_ingredients}

\begin{itemize}
	\item 600 g flour
	\item 300 ml water
	\item 100 g melted butter
	\item 1 packet of yeast
	\item 1 tbl spoon of sweetener (sucralose/stevia) or 2 tbl spoons of sugar
	\item smidge of salt
\end{itemize}
\subsubsection{Procedure}%
\label{ssub:classic_bread_procedure}

Start by melting the butter, kill the flame when about half of it is melted then swirl that shit around until the remaining knobs turn liquid. Leave it on the side.
Weigh out your flour, add yeast, salt, sugar, grab a wooden spoon and give it all a stir.\par

Check the butter's temperature, if it don't hurt your finger, chuck it into the flour mix. While stirring, pour all the water into the mix. At some point you'll have to stop using the spoon and get your hands dirty. Keep mixing it, and feel free to add flower should the dough still be sticky. Once it gets to the point where it's \textit{just} sticking to your fingers, transfer it onto a floured wooden board and get kneading. Knead that shit for good 5-10 minutes.\par

Once done, transfer it back into the bowl, cover with a cloth and let it rise for 1-3 hours depending on weather conditions.\par

Once it has at least doubled in size, make a couple of equally sized loaves, and transfer hem onto an oven tray, cover again and let it rise for another half hour.\par

When time's up, fire up the oven to 230 degrees C on fan mode. When ready, shove that shit in for 25-30 minutes. Done? Eat.
% fix the degrees in the document
\subsection{Czech Bread}%
\label{sub:czech_bread}

\subsubsection{Ingredients}%
\label{ssub:czech_bread_ingredients}
This amount makes one standard sized loaf to serve 4 ppl in one sitting

\begin{itemize}
	\item 600 g flour
	\item 300 ml full fat milk (room temp)
	\item 1 packet of yeast
	\item 1 tbl spoon of sweetener (sucralose/stevia) or 2 tbl spoons of sugar
	\item 1 large onion
	\item a chunk of butter, idk how much, i just cut, the more the better really
	\item 1 bottle of Pilsner Urquell
\end{itemize}
\subsubsection{Procedure}%
\label{ssub:czech_bread_procedure}

Open the bottle of PU and start drinking. Find a mug, a bowl, a small pot, frying pan and a chopping board you'll need those.\par

Ditch the hopefully warmed up milk into the mug and chuck the yeast on top, stirr a bit and let sit. Chuck the flour into a large bowl, and chuck the butter into a tiny pot and melt it. Don't actually let it burn. When it's \textit{just} melted, take it off the heat and take a big large sip of PU, you deserve it.\par

Get your chopping board, and start chopping up the onion into tiny little cubelettes. Chuck another chunk of butter into a frying pan, and get frying dat delicious onion. When it's just foamed up and slightly brown, take it off the heat. Take another gulp of beer.\par

By now the yeast should of worked its might and magic, and the melted butter in your other pan cooled down a bit, so chuck both guys into the bowl with flour, grab yourself a wooden spoon and get stirring, making sure all of the flour has been covered and a neat sticky dough has started to form.\par

Some minutes in, the onion has hopefully cooled down, so chuck that shit into the dough. At this point you better take a few big gulps of your beer, cos you gonna get your hands dirty boy. Work those onions into the dough, and knead it until it's either just or no longer sticking to the sides of the bowl. When it does, grab some extra flour and chuck it onto your chopping board, and transfer the dough onto it. Now you can knead it extra hard.\par

After a couple of minutes of some heavy kneading, either clean the bowl, or get another one. Grab a little chunk of butter (or just uravel the pack a bit) and smear it over the sides of the bowl. Once done, transfer the kneaded dough back into the bowl. Cover it with a dish cloth and let it sit in some place that's warm.\par

Now you gonna wait, so you might as well finish that beer. At 30 degrees C, the dough needs about an hour - an hour and half to rise like lazarus. At lower temps you might have to wait substantially longer. Do not leave it in a place above 35 you brute.\par

You managed to wait? great, check the dough, it should have \textit{at least} doubled up in size, good job! feel free to tear into it and give it a taste, it should taste tasty, like a tasty dough sohuld taste.\par

Turn your oven on, onto the fan function, and set the temp to 190 C.\par

Prep yourself a baking tray, cover the bottom in butter (same method as for the bowl above) and transfer the loaf onto it. If you're feeling insecure or want a slightly denser texture, feel free to knead the dough a bit, but let it rise another 20-30 minutes again (in the tray). Make some fancy incisions across the top of the loaf so it can bloom a bit in the oven. \par

In a second baking tray (or some heat resistant cup/mug/bowl/sacrificial vessel) put like 200ml of water and leave it at the bottom of the oven. When the oven reaches 190, place the tray with bread in the middle, and set the timer for 90 minutes.\par

Uhh...eat?
\subsection{Focaccia}%
\label{sub:focaccia}

\subsubsection{Ingredients}%
\label{ssub:focaccia_ingredients}
This amount makes one standard sized loaf to serve 4 ppl in one sitting

\begin{itemize}
	\item 500 g flour
	\item 200 ml water
	\item 150 ml olive oil
	\item 1 packet of yeast
	\item 2-4 cloves of garlic
	\item some black pepper
	\item some salt
	\item rosemary
	\item teaspoon of honey
	\item a bottle of moretti
\end{itemize}
\subsubsection{Procedure}%
\label{ssub:focaccia_procedure}

Open the bottle of Moretti and start drinking.\par

Fine chop the garlic, rosemary, crush the salt and pepper, grab about half of the olive oil and chuck it all on the frying pan. Stop just before the garlic browns. Leave the shit on the side to cool down, and take  a big sip of your bev. \par

Grab the flour, yeast, water, honey and the rest of the olive oil and mix that shit in a large bowl. You'll have to get your hands dirty, so you better take a big gulp of your Moretti now. Knead the everloving shit out of the dough. While you're doing this, hopefully the contents of the pan have cooled down sufficiently, because you're about to pour about a half of it into the mix. Don't blame me for any degree burns you might get. Knead it some more. \par

When done kneading, keep the dough in a bowl, cover it with a towel and leave to rise in a warm place for an hour or two. Finish drinking your beer or something. \par

Once risen, get your baking tray and oven ready. Preheat the oven on fan mode to 230 C. While that shit is preheating, baste the baking tray with some olive oil. Flop the dough onto the tray, shape it however you want, preferably a square, and poke some holes with your fingers on top. Pour the rest of the frying pan mix on top, spread it all over with a brush or something. Once done, shove it in the oven and bake for about 20 minutes. \par

Eat.
\section{Potato dishes}%
\label{sec:potato_dishes}

\subsection{Halusky}%
\label{sub:halusky}

Aka potato dumplings, gnocchi, whatever. This is the slovak version. This recipe makes for 3-4 servings.
\subsubsection{Ingredients}%
\label{ssub:halusky_ingredients}
\begin{itemize}
	\item 300 g potatoes
	\item 300 g flour
	\item teaspoon of salt
\end{itemize}
\subsubsection{Procedure}%
\label{ssub:halusky_procedure}


Find a cheese grater and grate the potatoes with skin on (use the medium/large sized holes, otherwise you'll be grating forever). When done, try to squeeze out about half of all the water content. Add the flour, and mix it all together with your hands.\par

Find a medium pot and start boiling some water. While the water is boiling, transfer a part of your dough onto a floured board and create a thin roll (two fingers wide suffices). Then with a knife or something, chop it up into tiny little dumplings (just google images halusky for size reference). When the water is boiling, chuck them in one by one. A single haluska is ready to be taken out when it will float on the surface, usually takes 4-6 minutes. This should give you enough time to continuously roll out and chop the next portion of the dough.\par

\subsection{Bryndzove halusky}%
This makes for 3 portions.
\label{sub:bryndzove_halusky}
\subsubsection{Ingredients}%
\label{ssub:bryndzove_halusky_ingredients}
\begin{itemize}
	\item Halusky \ref{sub:halusky}
	\item 300 g bryndza or feta cheese
	\item 150 g bacon 
	\item (optional) 100g sour cream
\end{itemize}
\subsubsection{Procedure}%
\label{ssub:bryndzove_halusky_procedure}
Fry up the bacon until crispy. Prep the cheese (and/or cream) into the bottom of a serving bowl, place the halusky straight out of the boiling water on top, it'll help melt the cheese. Stir it well, until the cheese covers all the halusky, pour some of the friend bacon/lard mixture on top. That's it.


\section{Desserts}%
\label{sec:desserts}

\subsection{Crepes}%
\label{sub:crepes}

While this portion may seem like it's for 3-4 people, let's face it, you're gonna eat it all on your own.

\subsubsection{Ingredients}%
\label{ssub:crepes_ingredients}
\begin{itemize}
	\item 250 g flour
	\item 6 eggs
	\item 50 g butter (plus extra for frying)
	\item 600 ml full fat milk (room temp)
	\item 60 g stevia
	\item 50 ml liquor of choice (rec: tatratea, chartreuse, benedictine)
	\item 10 g instant yeast
\end{itemize}

\subsubsection{procedure}%
\label{ssub:crepes_procedure}

Grab a big bowl and chuck all the ingredients inside except for the flour. Grab a whisk and start whisking. I highly recommend pouring the flour into the liquid mix bit by bit while you whisk instead of all at once to avoid clumping and shit. Whisk for good 10-15 minutes until supersmooth.

Heat up a pan, cut a little knob of butter, grab a ladleful of batter and spread it out evenly. Shake the pan a bit every now and then, when the mix is moving around relatively freely (takes about a minute or two) flip that shit like you're in the circus and cook the other side for slightly shorter amount of time. Rinse and repeat until you fry them all.
\end{document}
